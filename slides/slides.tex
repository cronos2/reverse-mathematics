\documentclass[10pt]{beamer}

\usepackage[utf8]{inputenc}
\usepackage{graphicx}
\usepackage{caption}
\usepackage[spanish]{babel}
\usetheme{metropolis}
\usepackage{appendixnumberbeamer}
\usepackage[spanish]{cleveref}

\usepackage[backend=biber, sorting=none]{biblatex}
\addbibresource{../bibliography/bibliography.bib}

\captionsetup[figure]{font=scriptsize, labelfont=scriptsize}

\newcommand{\ffigure}[2]{
  \begin{figure}[h!]
    \centering
    \includegraphics[width=0.6\textwidth]{images/#1.jpg}
    \caption{#2}
    \label{fig:#1}
  \end{figure}
}

\title{Matemática inversa: axiomas y teoremas y viceversa}
\date{\today}
\author{Ignacio Mas Mesa}
\institute{Universidad de Granada}

\begin{document}

\maketitle

\section{Introducción}

\begin{frame}[fragile]{¿Qué es la matemática inversa?}

  \begin{quote}
    When the theorem is proved from the right axioms, the axioms can be proved from the theorem.
  \end{quote}
  \flushright{--- Harvey Friedman}

  \vfill

  Algunos teoremas son equivalentes a ciertos axiomas.
\end{frame}

\begin{frame}[fragile]{Consideraciones}
  Cualesquiera dos fórmulas verdaderas son \textbf{equivalentes} desde un punto de vista lógico.

  \vfill

  Necesitamos un sistema axiomático más débil para formalizar esta idea.
\end{frame}

\section{Aritmética y aritmetización}

\begin{frame}[fragile]{El marco de trabajo}
  El sistema $Z_2$ de la aritmética de segundo orden.

  \begin{itemize}
  \item Axiomas básicos de la aritmética ($n + 1 \ne 0$, $n + 0 = n$, etc).
  \item Axioma de inducción
    \[ (0 \in X \land (n \in X \implies n + 1 \in X)) \implies \forall n (n \in X). \]
  \item Esquema de axioma de comprensión
    \[ \exists X \forall n (n \in X \iff \varphi(n)). \]
  \end{itemize}
\end{frame}

\begin{frame}[fragile]{Forma prenexa y jerarquía aritmética}
  Toda fórmula es equivalente a una fórmula en forma prenexa, i.e., con todos los cuantificadores apareciendo al inicio.

  La jerarquía aritmética permite asignar una medida de complejidad a una fórmula en función de cuántas veces alterna entre cuantificadores existenciales y universales.

  \[ \exists m (n = 2 \cdot m) \in \Sigma_1^0, \]
  \[ \forall n \exists m (n < m) \in \Pi_2^0, \]
  \[ (n + 2) \cdot k > m - 3  \in \Sigma_0^0 = \Delta_0^0 = \Pi_0^0, \]
\end{frame}

\section{Computabilidad}

\begin{frame}[fragile]{Funciones primitivas recursivas y parciales recursivas}
  \begin{itemize}
  \item Funciones iniciales: función cero, función sucesor y proyecciones.
  \item Composición.
  \item Recursión primitiva (codifica \texttt{reduce} y los bucles \texttt{for}).
  \end{itemize}

  La función de Ackermann es computable pero no primitiva recursiva

  \[
    A(m, n) :=
    \begin{cases}
      n + 1 & \text{si } m = 0, \\
      A(m - 1, 1) & \text{si } m > 0 \text{ y } n = 0, \\
      A(m - 1, A(m, n - 1)) & \text{en caso contrario.}
    \end{cases}
  \]

  Hace falta incluir el operador $\mu$ de minimización o búsqueda no acotada.
\end{frame}

\begin{frame}[fragile]{Máquinas de Turing}
  Tienen su propio formalismo matemático, pero conceptualmente representan una máquina, con un programa pregrabado, que puede leer y escribir sobre una cinta infinita, para lo cual se ayuda de estados.

  \ffigure{turing-machine}{Diagrama de una máquina de Turing}

  Son equivalentes en \emph{poder} a las funciones parciales recursivas, pero más \emph{intuitivas} que estas o que el cálculo $\lambda$. Constituyeron la base de los primeros ordenadores y aún hoy se identifican semejanzas.
\end{frame}

\begin{frame}[fragile]{Tesis de Church-Turing}
  Toda función para la cual existe un procedimiento intuitivamente efectivo para calcular sus valores es computable.
\end{frame}

\begin{frame}[fragile]{Enumerando máquinas de Turing}
  Dada una máquina de Turing, podemos asignarle un único número identificador de forma computable. Dado el número, podemos recuperar la máquina de Turing también de forma computable.

  Turing advirtió la importancia de esto: existe una máquina \alert{universal} de Turing. Este descubrimiento ayudó a establecer el paradigma del programa almacenado en disco.

  Existen problemas que no pueden ser resueltos por una máquina de Turing, como el \alert{problema de la parada}.
\end{frame}

\begin{frame}[fragile]{Reducibilidad, oráculos y grados de Turing (I)}
  La \alert{reducibilidad} nos permite comparar la dificultad relativa de dos conjuntos o problemas. Estudiamos las reducciones muchos-a-uno y las reducciones de Turing.

  Una máquina de Turing con \alert{oráculo} puede hacer preguntas de problemas que no se pueden resolver y usar las respuestas para resolver problemas aún más difíciles.

  Se puede generalizar el problema de la parada para obtener la operación \alert{salto de Turing}.
\end{frame}

\begin{frame}[fragile]{Reducibilidad, oráculos y grados de Turing (II)}
  La reducibilidad induce una relación de equivalencia sobre los conjuntos que da lugar a los \alert{grados de Turing}.

  El teorema de Post relaciona íntimamente la estructura de los grados de Turing con la de la jerarquía aritmética. En particular:

  \begin{itemize}
  \item $\Sigma_1^0$ es lo mismo que recursivamente enumerable,
  \item $\Pi_1^0$ es lo mismo que co-recursivamente enumerable,
  \item $\Delta_1^0$ es lo mismo que computable.
  \end{itemize}
\end{frame}

\section{Subsistemas de $Z_2$}

\begin{frame}[fragile]{El subsistema base: RCA$_0$}
  \begin{itemize}
  \item Axiomas básicos de la aritmética ($n + 1 \ne 0$, $n + 0 = n$, etc).
  \item Restringe el axioma de inducción a fórmulas $\Sigma_1^0$.
  \item Restringe el esquema de axioma de comprensión a fórmulas $\Delta_1^0$ (computables).
  \end{itemize}

  Permite definir $\mathbb{N}$, así como $\mathbb{Z}$ y $\mathbb{Q}$ por clases de equivalencia.

  Los números reales son sucesiones de racionales de Cauchy con ciertas propiedades adicionales.
\end{frame}

\begin{frame}[fragile]{WKL$_0$}
  El lema débil de König asegura que todo árbol binario infinito debe tener una rama infinita.

  El sistema WKL$_0$ añade el lema débil de König a los axiomas de RCA$_0$.

  Es propiamente más fuerte que RCA$_0$.
\end{frame}

\begin{frame}[fragile]{ACA$_0$}
  El sistema ACA$_0$ añade a los axiomas de RCA$_0$

  \begin{enumerate}
  \item el axioma de inducción de segundo orden,
  \item comprensión aritmética.
  \end{enumerate}

  Es propiamente más fuerte que WKL$_0$.
\end{frame}

\section{Resultados}

\begin{frame}[fragile]{Equivalencias en RCA$_0$}
  \begin{itemize}
  \item RCA$_0$ demuestra el teorema de Bolzano,
  \item En RCA$_0$ son equivalentes el lema débil de König y el teorema de Heine-Borel,
  \item En RCA$_0$ equivalen
    \begin{enumerate}[(i)]
    \item comprensión aritmética,
    \item teorema de Bolzano-Weierstra{\ss},
    \item teorema de la convergencia monótona,
    \item criterio de convergencia de Cauchy y
    \item existencia de supremo para sucesiones acotadas.
    \end{enumerate}
  \end{itemize}
\end{frame}

\section{Conclusiones y trabajo futuro}

\begin{frame}[fragile]
  La matemática inversa es una potente herramienta para el estudio fundacional de las matemáticas, que permite comparar la \textit{fuerza relativa} de dos teoremas formalizando esta noción.

  \vfill

  En caso de continuar con el proyecto podríamos considerar estudiar los sistemas ATR$_0$ y $\Pi_1^1$-CA$_0$, propiedades de los modelos de los sistemas que hemos escogido o profundizar en la estructura de los grados de Turing.
\end{frame}

\section{Ejemplo de matemática inversa: WKL$_0$ y el teorema de Heine-Borel}

\begin{frame}[fragile]{WKL$_0$ implica el teorema de Heine-Borel (I)}
  Sea $\{(a_i, b_i): i \in \mathbb{N}\}$ una sucesión de intervalos abiertos que recubre el intervalo $[0, 1]$.

  \ffigure{non-covered-intervals}{Construcción de $T$}

  Construimos un árbol binario $T$ en el que incluimos los intervalos $[\frac{m}{2^n}, \frac{m + 1}{2^n}]$ si dicho intervalo no está completamente recubierto $(a_1, b_1), \ldots, (a_n, b_n)$.
\end{frame}

\begin{frame}[fragile]{WKL$_0$ implica el teorema de Heine-Borel (II)}
  Para cada $x \in [0, 1]$, existe $i$ tal que $x \in (a_i, b_i)$ y $n$ suficientemente grande tal que

  \[ x \in \left[\frac{m}{2^n}, \frac{m + 1}{2^n}\right] \subset (a_i, b_i). \]

  Por tanto, $T$ no puede tener ramas infinitas, luego es finito. Entonces podemos encontrar un subrecubrimiento de $[0, 1]$

  \[ (a_1, b_1), \ldots, (a_r, b_r) \]

  para un $r$ apropiado.
\end{frame}

\begin{frame}[fragile]{El conjunto de Cantor}
  Para la demostración del recíproco haremos uso del conjunto de Cantor, $C$.

  \ffigure{cantor-set}{Conjunto de Cantor}

  Cada punto del mismo se puede identificar con una rama infinita del árbol binario completo.
  Además, la sucesión de intervalos $(\frac13, \frac23), (\frac19, \frac29), (\frac79, \frac89), (\frac{1}{27}, \frac{2}{27}), \ldots$ recubre $[0, 1] \setminus C$.

\end{frame}

\begin{frame}[fragile]{El teorema de Heine-Borel implica WKL$_0$ (I)}
  Asociaremos a cada vértice del árbol binario completo un intervalo abierto de forma que recubran $C$.

  \ffigure{covering-intervals}{Intervalos recubridores de $C$}

\end{frame}

\begin{frame}[fragile]{El teorema de Heine-Borel implica WKL$_0$ (II)}
  Sea $T$ un árbol binario sin ramas infinitas. Consideramos los intervalos asociados a sus hojas caídas, que recubren $C$.

  \ffigure{fallen-leaves}{Hojas caídas de $T$}


  Si unimos esta sucesión de intervalos a los que recubren $[0, 1] \setminus C$, obtenemos un recubrimiento de $[0, 1]$ por abiertos.
\end{frame}

\begin{frame}[fragile]{El teorema de Heine-Borel implica WKL$_0$ (III)}
  Por el teorema de Heine-Borel, podemos quedarnos con un subrecubrimiento finito. Sin embargo, ninguno de los intervalos que recubren $[0, 1] \setminus C$ contiene ningún elemento de $C$. Por tanto, existe un número finito de intervalos asociados a hojas caídas que recubre $C$.

  \vfill

  Por la construcción de los intervalos, esto significa que $T$ tiene un número finito de hojas caídas y por tanto es finito.
\end{frame}

\begin{frame}[standout]
  Preguntas

  \vfill

  \normalsize \url{https://github.com/cronos2/reverse-mathematics}
\end{frame}

\appendix

\begin{frame}[fragile]{Referencias}
  \Cref{fig:turing-machine}: \textit{Turing machine} por Nynexman4464. \url{https://commons.wikimedia.org/wiki/File:Turing_machine_2b.svg}

  \vfill

  \Cref{fig:non-covered-intervals}, \Cref{fig:cantor-set}, \Cref{fig:covering-intervals}, \Cref{fig:fallen-leaves}: Stillwell, J. (2019) \textit{Reverse Mathematics: Proofs from the Inside Out}. Princeton University Press (pp. 137, 138, 141), fig.
\end{frame}

\begin{frame}[allowframebreaks]{Bibliografía}
  \nocite{*}
  \printbibliography[heading=none]
\end{frame}

\end{document}

%%% Local Variables:
%%% mode: latex
%%% TeX-master: t
%%% End:

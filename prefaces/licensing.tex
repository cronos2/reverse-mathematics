\documentclass['../proyecto.tex']{memoir}

\begin{document}

\thispagestyle{empty}

\noindent\rule[-1ex]{\textwidth}{2pt}\\[4.5ex]

Yo, \textbf{\AuthorName}, alumno del Doble Grado en
Ingeniería Informática y Matemáticas de la \textbf{\FacultyOne de la
  \University}, con DNI 71967506T, autorizo la ubicación de
la siguiente copia de mi Trabajo Fin de Grado en la biblioteca del
centro para que pueda ser consultada por las personas que lo deseen.

\vspace{6cm}

\includegraphics[width=3.5cm]{signature.png} \\
\noindent Fdo: Ignacio Mas Mesa

\vspace{2cm}

\begin{flushright}
  \Location, \Time
\end{flushright}

\newpage

\thispagestyle{empty}

\noindent\rule[-1ex]{\textwidth}{2pt}\\[4.5ex]

D. \textbf{\MainProf}, profesor del
\Department de la \University.

\vspace{0.5cm}

\textbf{Informa:}

\vspace{0.5cm}

Que el presente trabajo, titulado \textit{\textbf{\ProjectTitle}}, ha
sido realizado bajo su supervisión por \textbf{\AuthorName}, y
autorizamos la defensa de dicho trabajo ante el tribunal que
corresponda.

\vspace{0.5cm}

Y para que conste, expide y firma el presente informe en \Location a \Time

\vspace{1cm}

\textbf{El director:}

\vspace{5cm}

\begin{minipage}{0.45\linewidth}
  \begin{center}
    %\includegraphics[width=5cm]{}
    \textbf{\MainProf}
  \end{center}
\end{minipage}

\chapter*{Agradecimientos}

En primer lugar, quiero agradecerle a toda mi familia el haberme apoyado siempre en todas mis andaduras, incluso cuando la experiencia les decía que mis elecciones podían no ser las óptimas. Gracias en especial a mi madre, mi padre y mis abuelos por haberme proporcionado una educación que me permitirá desarrollarme con mayor facilidad en la etapa de la vida que ahora inicio y por haberme inculcado los ideales de esfuerzo, dedicación y búsqueda del conocimiento. Cualquier desviación por mi parte de estos es sólo atribuible a mis propios errores.

Por otro lado, debo agradecer a la que ha sido mi familia durante estos seis años, tanto a los que han estado desde el principio (Croqueta, S, Fua, Óxido) como a los que llegaron más tarde (Petaca, Linguini, Geyper, Spock, Tallo). A Caracas, por haberme enseñado que por lo general los absolutos no tienen cabida y por ser una fuente constante de inspiración. A Mojo, porque sin él probablemente nunca habría llegado a Granada y mi vida sería completamente distinta; y por ser una persona noble como ninguna. A Gavilán, por haber compartido algunos de los mejores años de nuestras vidas juntos y por seguir siendo un enigma para mí. A Fidel, por las innumerables cachimbas, las veces que ha jugado Blitzcrank en vez de Zed, las horas dedicadas al ajedrez, las matemáticas, la física o la programación; e incluso por las correciones a este trabajo. Y a Lucía, por enseñarme que el amor y la amistad deben ir de la mano y que ``aufabaguera'' significa albahaca.

Gracias en general a todas las personas que de un modo u otro han pasado por mi vida y han influido en mi modo de ver el mundo: profesores, conocidos, camareros, panaderos, bibliotecarios, músicos, cineastas, matemáticos o lingüistas. Gracias en particular a toda la gente buena de Granada que ha ayudado a que mi estancia en la ciudad fuera un poco mejor. Este trabajo es también, en parte, suyo.

Agradecer también a la comunidad del software libre, que tanto aporta a la sociedad y que tan frecuentemente es olvidada.

Por último, pero no por ello menos importante, gracias a Andrew L., cuyo anónimo comentario en Facebook me descubrió el mundo de la matemática inversa que, de otro modo, habría permanecido ignoto para mí.

\thispagestyle{empty}

\vspace{1cm}

\newpage

\end{document}

\documentclass[../main.tex]{memoir}
\begin{document}
\thispagestyle{empty}

\begin{center}
  {\large\bfseries \ProjectTitleEsp}\\
\end{center}

\begin{center}
  \AuthorName\\
  \vspace{0.7cm}
  \noindent{\textbf{Palabras clave}: matemática inversa, axiomas, teoría de la computabilidad, jerarquía aritmética, equivalencia lógica.}\\

  \vspace{0.7cm}
  \noindent{\textbf{Resumen}}\\
\end{center}

La rama de la matemática inversa tal y como la conocemos hoy en día nació en 1975 con el artículo de Harvey Friedman ``Some Systems of Second Order Arithmetic and Their Use'' \cite{friedman}. En los años que siguieron a dicho artículo la rama se convirtió en un programa en el que se invirtió una gran cantidad de esfuerzo por parte de diversos investigadores con la intención de desarrollar, mejorar y establecer sus elementos básicos del discurso y principales características definitorias, así como a la profundización en el estudio de sus consecuencias y resultados más interesantes. En 2009, Stephen G. Simpson publicó un tratado de carácter enciclopédico sobre el estado actual de la rama, ``Subsystems of Second Order Arithmetic'' \cite, que hoy en día sigue siendo la referencia de facto sobre la materia y que ha sido frecuentemente consultado y referenciado para la redacción del presente documento. Una introducción más accesible a los principales conceptos presentados en el libro de Simpson fue publicada en 2018 por John Stillwell: ``Reverse Mathematics: Proofs from the inside out'' \cite{stillwell}. Este libro ha servido como fuente de inspiración para los aspectos más didácticos de este trabajo.

En su forma actual, la matemática inversa se encarga de estudiar la relación entre axiomas y teoremas. Durante la mayor parte de la historia de las matemáticas, los axiomas han sido considerados una herramienta fundacional sobre la que construir el resto de resultados matemáticos mediante un proceso deductivo con el que los matemáticos hoy en día están totalmente familiarizados. Aunque existen algunos ejemplos históricos importantes de estudio de los axiomas en sí mismos, entre los cuales destacan el quinto postulado de Euclides así como el axioma de elección o la hipótesis del continuo en la teoría de conjuntos, la matemática inversa actual estaba sin explorar hasta que Friedman presentó las ideas y herramientas necesarias para la concreción de sus objetivos. En su versión más elemental, la idea clave de la matemática inversa es que la relación entre ciertos axiomas y teoremas no es únicamente de implicación sino de equivalencia. Esto es, en palabras del propio Harvey Friedman: ``Cuanto el teorema es demostrado a partir de los axiomas correctos, los axiomas pueden ser demostrados a partir del teorema''. Por supuesto, esta idea requiere de clarificación y formalización de modo que pueda ser útil para los matemáticos y este será precisamente nuestro principal objetivo en los capítulos venideros.

El capítulo de introducción expone las ideas y conceptos básicos con los que trabajaremos posteriormente de cara a cumplir nuestro objetivo. El principio más destacable de este capítulo es la formalización de la equivalencia a la que hacíamos referencia previamente entre axiomas y teoremas. Para ello, necesitaremos un sistema axiomático base en el que construir la demostración de la equivalencia. La forma clásica de proceder sería asumir el sistema base así como el axioma en el que estamos interesados para demostrar el teorema. La matemática inversa recibe su nombre por la otra parte de la demostración, en la que se asumen el sistema base y el teorema para demostrar el axioma. Es importante tener en cuenta que el sistema base que buscamos debe ser suficientemente expresivo como para poder enunciar el axioma y el teorema (e.g., si se quiere demostrar el teorema de Weierstra{\ss} debemos definir en primer lugar qué se entiende por función continua en el sistema base), pero también tan débil que ni el axioma en el que estamos interesados ni el teorema puedan ser demostrados en él.

Después de presentar las ideas y definiciones fundamentales, continuamos con algunos conceptos más avanzados y pasamos a los detalles. Para empezar, realizamos las definiciones necesarias para llegar hasta el concepto de jerarquía aritmética, que nos será de vital importancia para el trabajo posterior y por el que podemos definir una medida de complejidad de ciertas formulas de segundo orden de nuestro lenguaje. También definimos el sistema de la aritmética de segundo orden así como el subsistema en el que estaremos más interesados, a saber, \rca. Por último, demostramos cómo la función emparejadora es capaz de subsumir la estructura de los enteros y los racionales dentro de los naturales y presentamos una definición sintética pero práctica de número real en \rca.

El capítulo siguiente está dedicado a teoría de la computabilidad, que puede parecer una digresión en primera instancia. Empezamos por presentar la aproximación clásica de la teoría de la recursión, en la cual la computabilidad se define inductivamente para una clase reducida de funciones iniciales y para ciertas combinaciones de ellas. También presentamos la alternativa más conocida de las máquinas de Turing y esbozamos una demostración de su equivalencia (en el sentido de que ambas definiciones de computabilidad conducen a la misma clase de funciones). Concluimos esta sección enunciando la tesis de Church-Turing, que nos permitirá proceder de forma más laxa durante el resto del capítulo, pero también confiar en que nuestros resultados pueden ser respaldados por argumentos rigurosos en caso de necesitarlo.

Pasamos entonces a explorar el concepto de enumerabilidad recursiva y presentamos algunos ejemplos y propiedades, descubriendo que las máquinas de Turing (o las funciones parcialmente recursivas, si se quiere) pueden ser enumeradas e incluso demostrando la existencia de una máquina de Turing universal. De este modo encontramos un ejemplo natural de conjunto recursivamente enumerable pero no computable. De hecho, este ejemplo concreto está muy íntimamente relacionado con el problema de la parada, lo cual nos sirve para ligar nuestros desarrollos teóricos con algunos de los aspectos más conocidos de la teoría de la computabilidad.

Más tarde introducimos el concepto de reducción, en particular las reducciones muchos a uno y las reducciones de Turing. Esto nos permite explorar las máquina de Turing con oráculo, la generalización de nuestro trabajo previo a la \textit{computabilidad relativa} y los grados de Turing. Para culminar, todos los aspectos de la teoría de la computabilidad desarrollados hasta el momento nos permiten enunciar el teorema de Post, que relaciona de manera profunda los conceptos estudiados en este capítulo con la jerarquía aritmética. En particular, sin ser muy precisos, descubrimos que la jerarquía aritmética es a los grados de Turing lo que \rca\ a la computabilidad (o los objetos computables). En otras palabras, \rec\ es lo mismo que computable y \re\ es equivalente a la enumerabilidad recursiva.

Tras esto, presentamos de nuevo el sistema \rca\ y algunas de las construcciones en las que estaremos más interesados. Específicamente, definimos en \rca\ las sucesiones de racionales y de números reales y también las funciones continuas. También en este capítulo se introducen dos subsistemas de la aritmética de segundo orden adicionales: \wkl\ y \aca.

Para \wkl\ definimos en \rca, como de costumbre, los conjuntos finitos, sucesiones finitas, árboles binarios y caminos infinitos en árboles. A continuación presentamos la versión débil del lema de König y definimos el sistema \wkl\ consistente en los axiomas \rca\ junto con la versión débil del lema de König. Finalmente, las fórmulas aritméticas y el esquema de axioma de inducción de segundo orden habían sido ya introducidos en el primer capítulo, por lo que el sistema \aca\ con los axiomas de \rca, el esquema de axioma de inducción de segundo orden y la comprensión aritmética es definido sin dificultad. En este capítulo también exploramos otros aspectos de dichos subsistemas como la estructura de sus $\omega$-modelos en términos de teoría de la computabilidad así como sus partes de primer orden, i.e., qué fórmulas de primer orden son capaces de demostrar.

Por último, el capítulo final logra el objetivo que nos habíamos fijado al inicio del trabajo y pone en práctica toda la teoría desarrollada para mostrar algunos ejemplos concretos de matemática inversa. En particular, demostramos que \wkl\ y \aca\ son cada uno equivalentes a gran cantidad de resultados clásicos del análisis en \rca, como por ejemplo el teorema de Heine-Borel o el de Bolzano-Weierstra{\ss}. Así, demostramos cómo la matemática inversa puede ser usada como una herramienta efectiva para determinar la fuerza de distintos teoremas cuando se comparan. Por ejemplo, el hecho de que el teorema de la convergencia monótona sea equivalente a \aca\ y \aca\ más fuerte que \wkl\ significa que es \textit{más fuerte} que la compacidad del intervalo $[0, 1]$, también conocido como el teorema de Heine-Borel.

Un capítulo adicional para conclusiones y trabajo futuro identifica las principales ideas y resultados desarrollados en los capítulos anteriores y muestra cómo existe una plétora de temas relacionados que no se estudian en este documento.

\newpage
\end{document}

%%% Local Variables:
%%% mode: latex
%%% TeX-master: "../main"
%%% End:
